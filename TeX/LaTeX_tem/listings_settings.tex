\usepackage{fontspec}
% 设置 Menlo 字体
\setmonofont{Menlo}
\usepackage{fancyvrb}
\usepackage{xcolor}
\usepackage{listings}

% \definecolor{string}{HTML}{067D17}
% \definecolor{comment}{HTML}{8C8C8C}
% \definecolor{keyword}{HTML}{0033B3}
% \definecolor{class_field}{HTML}{871094}

\lstset{breaklines}
%这条命令可以让LaTeX自动将长的代码行换行排版
\lstset{extendedchars=false}
%这一条命令可以解决代码跨页时,章节标题,页眉等汉字不显示的问题
\lstset{escapeinside={(*}{*)}}

\lstset{
    basicstyle=\small\ttfamily\heiti,
    numbers=left,
    numberstyle=\scriptsize\fontspec{Menlo}, % 使用 Menlo 字体
    stepnumber=1,
    numbersep=8pt,
    frame=leftline,
    xleftmargin=2em, % 调整代码块的左边界
    framexleftmargin=0pt, % 调整边框的位置
    breaklines=true,
    % postbreak=\mbox{\textcolor{red}{$\hookrightarrow$}\space},
    % keywordstyle=\bfseries\color{keyword},          % keyword style
    % commentstyle=\heiti\color{comment},       % comment style
    % stringstyle=\color[HTML]{067D17},
    showstringspaces=false,
    % string literal style
    % escapeinside={\%*}{*)},            % if you want to add LaTeX within your code
    % morekeywords={}               % if you want to add more keywords to the set
}