\documentclass[10pt,UTF8]{book} %% ctexart

\title{\textbf{程序设计基础 (Java)}}
\author{钱锋\thanks{Email: strik0r.qf@gmail.com}${}^,$\thanks{
    西北工业大学软件学院, School of Software, Northwestern Polytechnical University, 西安 710072
}}

\usepackage{ctex}
\usepackage{graphicx}
\usepackage[toc]{multitoc}
\usepackage{booktabs}
\usepackage{longtable}
\usepackage{amsthm, amssymb, amsmath, mathrsfs, mhchem}
\usepackage{tikz}
\usetikzlibrary{decorations.markings, angles, quotes}
\usepackage{pgfplots}
\usepackage{tikz-3dplot}
\usepackage{extpfeil}
\usepackage{diagbox}
\usepackage{float}
\usepackage{hyperref}
\hypersetup{hidelinks,
    colorlinks = true,
    allcolors = black,
    pdfstartview = Fit,
    breaklinks = true}
\usepackage{caption}
\usepackage{enumitem}
\usepackage{siunitx}

\input{LaTeX_tem/fancyhdr_settings.tex}
\input{LaTeX_tem/titlesec_settings.tex}
\input{LaTeX_tem/geometry_settings.tex}
\input{LaTeX_tem/mdframed_settings.tex}
\input{LaTeX_tem/listings_settings.tex}

\usepackage{smartdiagram}

\begin{document}

\input{LaTeX_tem/theoremstyles.tex}
\pagestyle{empty}
\begin{titlepage}
    \thispagestyle{empty}
    \centering
        \vspace*{2cm}
        \includegraphics[width=0.5\textwidth]{pic/npu_2.png}\par
        \vspace{1em}
        \includegraphics[width=0.5\textwidth]{pic/npu_1.png}\par
    \vspace{1em}
        \begin{center}
            \Huge \heiti \textbf{程序设计基础 (Java)}

            Fundamentals of Computer Programming
        \end{center}
        \vspace{17em}
        \begin{center}
        \songti
        \kaishu 软件学院 \, \heiti\textbf{钱锋} \quad \songti 编
        \vspace{0.5em}

    \today
    \end{center}
\end{titlepage}
\cleardoublepage
\maketitle
\cleardoublepage

\frontmatter
\newpage
\pagestyle{plain}
\makeatother

% 谨以此书, 敬伟大的数学家、伟大的数学教育家

% \begin{center}
%     Vladimir A. Zorich.
% \end{center}

% 这是一本抄来的数学书, 这本书的作者不是那个在海边发现贝壳的人, 只是一个沿着前人的足迹在展览馆里给贝壳拍照的后来人.

% 愿你在美妙的数学旅程中, 发现梦想, 追求真理和自由!


% \input{丛书前言.tex}


% \chapter*{第一版前言}
% \markboth{程序设计基础 (Java)}{第一版前言}
% \addcontentsline{toc}{chapter}{第一版前言}
% % 设置前言标题页的页码格式为empty, 即无页眉页脚
% \thispagestyle{empty}

% 本书是西北工业大学软件学院开设的《面向对象编程与设计》课程的先导参考资料,
% 旨在用尽可能短的时间帮助大家入门 Java 程序设计, 以更快地跟上《面向对象编程与设计》课程
% 直接以 Java 的面向对象特性为起点的教学进度.

% \begin{flushright}
%     \kaishu
%     钱锋

%     西北工业大学软件学院

%     2023 年 8 月
%     \songti
% \end{flushright}

\newpage
\thispagestyle{empty}

% 设置目录页的页码格式
\pagenumbering{roman} % 切换回罗马数字页码
\addtocontents{toc}{\protect\thispagestyle{empty}}
\pagestyle{plain}
{\tableofcontents}
\newpage
\thispagestyle{empty}
\cleardoublepage % 确保正文从奇数页开始


% 设置章节标题页的页眉和页脚为空白页样式
\makeatletter
\let\ps@plain\ps@empty
\makeatother

\mainmatter

\part{面向对象分析与设计}

\chapter{类与对象}

面向对象程序设计 (Object Oriented Programming, OOP)
就是使用对象进行程序设计. 本章与后续几章将以一个 GTD 个人事务管理系统为例
讨论 Java 的面向对象特性,
如果读者想要了解有关面向对象的相关内容, 可以参阅笔者编写的
《面向对象编程与设计》.

\section{面向对象的基本概念}

\subsection{类与对象}

\textbf{对象} (object) 是类的一个实例, 有状态和行为. 是属性 (数据、状态) 和处理属性的方法 
(行为、操作、服务) 的封装, 其中属性和方法作为一个不可分离的整体封装在一起, 
这也意味着方法是依赖于对象存在的, 没有游离于对象之外的方法.
一个对象的状态是由\textbf{数据域} (field) 及其当前的值来表示的,
对象的行为是由\textbf{方法} (method) 来定义的.

\textbf{类} (class) 是一个创建对象的模板, 它规定了该类中所有对象的属性和方法, 
对象是类的实例, 可以从一个类中创建多个实例.
如果我们想要通过类为一类事物建模, 最关键的步骤是从这一类事物的许多具体的
实例中抽象出主要特征.

我们可以通过类为我们平时生活中需要处理的工作/事务/任务 (task) 进行建模. 
一个任务可以由以下的信息所确定:
\begin{itemize}[itemsep=0pt]
    \item 对任务的简单描述 (description), 它指示了这一个任务大概将要
    做什么, 例如, “喂猫”、“完成本周的离散数学课程作业”、“观看 Strik0r 的新视频”
    或者 “阅读《深入理解计算机系统》”.
    \item 截止时间 (deadline), 它指示了一个任务在哪一个具体的时间点
    后将不再有效. 具体而言, 如果你的离散数学课在本周日的 23:30 截止提交,
    那么你必须在这个时间点前完成并提交, 否则将会影响你的平时成绩.
    \item 安排时间 (arrangement), 它指示了你决定大概在何时去做
    这一项任务, 要注意的是, 为任务所安排的时间和任务的截止时间是不同的,
    这就是说, 虽然你的离散数学作业在本周日的 23:30 才截止,
    但是你依然可以安排在今天、明天或者有效期内的任意一天来完成这项工作.
    \item 对任务的具体描述 (info), 它详细地记录了与任务有关的一些信息.
\end{itemize}
除此之外, 一个任务可能还涉及到 “委托人” (即你把这项工作分配给谁来完成)、
“优先级” (即这项任务应当被优先完成, 还是可以适当延迟, 或者干脆没有优先级)、
“标签” (这是你给任务设置的标记, 可以用于在信息管理系统中进行筛选)、
“前置任务” (即为了完成这项任务, 你必须先完成什么任务) 等.
我们现在只考虑简单描述 \lstinline|description|、截止时间 \lstinline|deadline|、
安排时间 \lstinline|arrangement| 和具体描述 \lstinline|description|,
考虑到数据管理的需求, 我们再为它增加一个唯一的标识数据域 \lstinline|id|.
那么一个任务实际上可以被一个包含有简单描述、截止时间和安排时间的
元组所唯一的确定, 我们抓住这一主要特征, 把这些任务抽象为一个类 \lstinline|Task|,
我们将会在下一节中讲到如何在 Java 中将其创建出来.

\begin{figure}[H]
    \centering
    % 图的内容
    \includegraphics*[width=0.45\textwidth]{pic/UML/Task.png}
    \caption{\lstinline|Task| 类的 UML 类图}
    \label{uml:class TaskManagement.Task}
\end{figure}

\subsection{类的定义}

Java 是纯面向对象的语言, 因此 Java 程序中的所有函数 
(我们在 Java 中习惯将其称为\textbf{方法}, method) 都必须放在\textbf{类} (class,
本书中常称为 Java class) 中来定义,
更彻底地说, Java 程序中的所有内容都必须在类中来组织.
类是 Java 程序的基本组成部分, 不管多大的 Java 工程, 都是由若干个 Java class 组织起来的.

在 Java 中使用 \lstinline|class|
关键字定义一个类, 具体的语法为
\begin{lstlisting}
(accessModifier) class className {

}
\end{lstlisting}
关于 Java class 的定义, 我们做一些说明:

首先, 在 \lstinline|class| 关键字前可以添加 
\lstinline|public| \textbf{访问修饰符} (access modifier), 我们会在后面详细介绍
各种访问修饰符的作用及其使用方法, 现在你只需要知道, 访问修饰符用于控制程序的其他部分对
这段代码的访问权限级别, Java 对访问权限的控制是封装性在 Java 中的一个具体实现.

因此, \lstinline|public class Hello {}| 表示我们定义了一个
访问权限为 \lstinline|public| 的类 \lstinline|Hello|, 而类中的所有元素的定义全部放在
类名后面的大括号里.

其次, 在一个 Java 源文件中你可以定义多个类, 
但是访问修饰符声明为 \lstinline|public| 的类 (这样的类称为\textbf{主类}) 最多
只能有一个 (可以没有), 并且如果有的话, 源文件的文件名必须与主类的类名相同, 
这就是我在前文中叫你把源文件命名为 \lstinline|Hello.java| 的原因.
如果你没有注意到我的提示,
把类 \lstinline|Hello| 声明在了其他源文件中, 那么你将会收到编译器的警告:
\kaishu 错误: 类 \lstinline|Hello| 是公共的,
应在名为 \lstinline|Hello.java| 的文件中声明. \songti

一般来说, 我们约定一个源文件中仅编写一个类 (除非有测试需求).

第三, Java 的类命名是有一定的规则的, 不合适的类命名会让人感到困惑, 进而
大大降低代码的可读性和可维护性, Java class 的命名规则为:
\begin{itemize}[itemsep = 0pt]
    \item 类名必须以字母开头, 后面可以是字母和数字的任意组合, 一般用名词来给类命名;
    \item 采用 \lstinline|Pascal| 命名规则, 即每一个英文字母的首字母大写;
    \item 类名不能使用 Java 保留字;
\end{itemize}


\begin{code}
    \lstinline|Task.java|
    \lstset{
        moreemph = {
            id, description, deadline, arrangement,
            info
        },
        emphstyle = \color{class_field},
    }
    \lstinputlisting[language=Java]{../src/cn/edu/nwpu/software/strik0r/TaskManagement/Task/Task.java}
\end{code}
\chapter{Java 语言概述}

\quad\quad 关于 Java 语言的历史、计算机基础等内容我们在本章不做赘述, 我们直接从 Java 语言
的一些基本使用开始. 本章介绍一个简单的 Java 程序 \lstinline|Hello.java|, 并介绍 Java
的一些基础语法和 Java 语言的特点.

\section{Hello Java World!}


\begin{example}[Hello, Java!]
    \label{helloJavaWorld}
    这是一个经典的 hello 程序, 我们基于它来介绍 Java 程序的基本结构. 请编写一个
    \lstinline|Hello.java| 的 Java 类文件, 在 IDE 中输入以下代码, 然后尝试运行它:
    % \lstinputlisting[language=Java]{../IdeaProjects/Hello/src/Hello.java}
    这段程序的执行结果是向控制台输出信息 \lstinline|Hello Java World!|.
    接下来, 我们将从类的定义、方法的定义和 \lstinline|main| 方法、
    Java 语句和 \lstinline|.| 运算符三个角度来解释这段代码.
\end{example}



\subsection{方法的定义和 \lstinline|main| 方法}

\subsubsection{方法的定义}

在 Java 中定义一个方法的语法为
\begin{lstlisting}
(accessModifier) (static) returnType methodName(arg1Type arg1, ...) {

}
\end{lstlisting}
其中 \lstinline|accessModifier| 和 \lstinline|static| 是可选的, 
\lstinline|returnType| 指的是该方法的返回类型, 
\lstinline|methodName| 是方法名,
\lstinline|argType| 是该方法接受的
参数的类型, \lstinline|arg| 是该方法接收的参数名 (我们往往将其称为\textbf{形式参数}, 
你只要把它理解为函数 $f(x)$ 里的 $x$ 就行了, 其中, $x$ 是一个 \lstinline|argType| 类型
的变量). 方法中需要执行的指令序列称为方法体, 放在方法声明后的大括号中.

\subsubsection{\lstinline|main| 方法的定义}

在 \lstinline|main| 方法的定义中, 我们使用了关键字 \lstinline|public| 
和 \lstinline|static|, 并且指定 \lstinline|main| 方法没有返回值 (\lstinline|void|).
现在你不需要去管形式参数 \lstinline|String[] args| 是什么 (老实说, 直到我在写这一章的时候
我依然不知道它是什么), 我们会在后面详细介绍.
你需要知道的是:
\begin{enumerate}
    \item \lstinline|main| 方法是 Java 应用程序的执行入口, 它的固定声明格式为
    \begin{lstlisting}
class className {
    public static void main(String[] args) {}
}
    \end{lstlisting}
    现阶段, 请你向八股文一样把它死背住, 随着我们学习的深入, 你会逐渐理解这里面每一个关键词的含义.
    
    \item 每一个类都可以有 \lstinline|main| 方法,
    即 \lstinline|main| 方法也可以写在非 \lstinline|public| 类中, 然后指定运行非
    \lstinline|public| 类, 那么此时程序的入口就将是该非 \lstinline|public| 
    类的 \lstinline|main| 方法.
    
    \item 需要注意的是, \lstinline|main| 方法的拼写是 main, 许多人会在不经意间错拼成 mian,
    因此当找了很久找不到错误时, 不妨看看你有没有犯这一个低级的拼写错误.
\end{enumerate}

% 对象名、变量名和方法名采用 \lstinline|camelCase| 命名规则, 即第一个单词全小写, 
% 第二个单词开始每个单词的首字母大写. 除此之外, 类名和方法名不能使用 Java 的关键字, 
% 也不能使用数字开头,
% 类名和方法名是大小写敏感的, 也就是说, \lstinline|method1| 和 \lstinline|Method1|
% 是两个不一样的方法.

\subsection{Java 语句和 \lstinline|.| 运算符}

\lstinline|System.out.println("Hello Java World!");| 是 \lstinline|main|
方法中唯一的指令, 它的作用是向控制台输出字符串 \lstinline|Hello Java World|,
在这里你不需要掌握太多, 需要你明确的是:

\begin{enumerate}
    \item 一个 Java 语句需要用 \lstinline|;| 结尾, 如果不加分号的话是会报错的,
    语句末尾缺少分号或者输入全角分号 \lstinline|: | 是初学者常犯的错误.
    
    \item Java 中利用大括号 \lstinline|{}| 来划分代码的结构, 先前提到的类体、方法体都是写在类声明、
    方法声明后的大括号中的, Java 中的大括号总是成对出现的, 请你养成成对输入大括号的习惯.
    关于大括号的次行风格和行尾风格我们不再单独介绍,
    我本人采用的是行尾风格 (听不懂吗?
    没有关系, 如果你不知道我在说什么的话, 你就按照我的示例代码来输入大括号就可以了).
    
    \item Java 是面向对象的语言, 因此我们通常用 \lstinline|object.method()| 来调用
    对象的方法, 用 \lstinline|object.attribute| 来访问对象的属性.
\end{enumerate}

\section{Java 的基础语法}

\subsection{Java 转义字符}

\begin{table}[H]
    \caption{Java 中的转义字符}
    \centering
    \begin{tabular}{l l p{6cm}}
        \hline
        转义字符 & 用途 & 备注 \\
        \hline
        \lstinline|\t| & 制表位 & 用于实现对齐 \\
        \lstinline|\n| & 换行符 & 将光标移动到下一行的开头 \\
        \lstinline|\"| & 双引号 \\
        \lstinline|\'| & 单引号 \\
        \lstinline|\r| & 回车   & 将光标移动到当前行的开头, 新输出的内容会顶掉原先输出的内容 \\
        \hline
    \end{tabular}
\end{table}

\subsection{Java 注释}

用于注解、说明、解释程序的文字就是注释 (comment), 
被注释的文字不会被 \textbf{Java 虚拟机} (Java Virtual Machine, JVM) 解释执行,
因此你可以放心地往代码中添加注释,
注释能有效地提高代码的可读性并
降低团队成员间的沟通成本, 养成良好的注释习惯是很重要的. 不写注释的代码就像从头到尾只有
逻辑符号和运算过程, 而没有任何文字说明的数学文献一样粗暴.

良好的编程习惯是\textbf{自顶向下的实现}, 你首先需要做一个顶层的设计和规划, 然后再编码阶段
先通过注释写出你的实现策略和实现思路, 然后再逐步用代码去实现你的想法. 你的实现思路可能是由
若干个子模块组成的, 对于这些子模块, 采用同样的自顶向下实现方法, 直到它们被分解地足够基本
以至于每一个模块只实现一个很简单的功能为止.

在 Java 中实现注释的方式有三种: 单行注释、多行注释和 javadoc 文档注释, 它们的使用方法如下:
\begin{lstlisting}
// Single line comment

/*
  Multi line comments
  System.out.println("This line won't be executed.");
  */

/**
 * javadoc comments
 * @javadocTag this comment will be compliered by javadoc tool.
 */
\end{lstlisting}
关于注释的使用, 我们说明以下几点:
\begin{enumerate}
    \item 每一个 \lstinline|*/| 都会被认为是多行注释的结束, 因此
    多行注释内不可嵌套多行注释, 所以不要简单地把一段代码用 \lstinline|/* */| 括起来
    就认为它们已经被全部注释掉了.
    \item 文档注释的注释内容可以被 JDK 提供的 javadoc 工具所解析, 生成一套以网页文件的形式
    体现的程序说明文档, 其中 \lstinline|@javadocTag| 是 javadoc 标签, 常见的 javadoc
    标签见下表所示:
    \begin{table}[H]
        \centering
        \caption{常用 Javadoc 标签}
        \begin{tabular}{p{2.5cm} p{5cm} p{7cm}}
            \hline
            \textbf{标签} & \textbf{描述} & \textbf{示例} \\
            \hline
            \lstinline|@param| & 用于描述方法的参数, 提供参数的名称和描述.  & 
            \begin{lstlisting}
/**
 * @param a 第一个加数
 * @param b 第二个加数
 */
            \end{lstlisting} \\
                              \hline
            \lstinline|@return| & 用于描述方法的返回值类型和意义.  & 
            \begin{lstlisting}
/**
 * @return 学生的姓名
 */
            \end{lstlisting} \\
                              \hline
            \lstinline|@throws| & 用于描述方法可能抛出的异常类型和原因.  & 
            \begin{lstlisting}
/**
 * @throws Exception
 */
            \end{lstlisting} \\
                              \hline
            \lstinline|@deprecated| & 标记方法或类已经过时, 提醒开发者不再使用.  & 
            \begin{lstlisting}
/**
 * @deprecated 使用新方法代替
 */
            \end{lstlisting} \\
                                    \hline
            \lstinline|@see| & 引用其他类、方法、字段或文档.  & 
            \begin{lstlisting}
/**
 * 查看其他类的文档
 * @see OtherClass
 */
            \end{lstlisting} \\
                           \hline
        \end{tabular}
        \label{tab:javadoc-tags}
    \end{table}
    \item 类、方法的注释, 要用 javadoc 文档注释来写, 非 javadoc 文档注释内容,
    往往是为了便于代码的维护, 即告诉读者这段代码的思路以及相关的注意事项.
\end{enumerate}

\section{Java 语言的特点}

Java 的设计者编写了一个白皮书\footnote{可以在 
\url{www.oracle.com/technetwork/java/langenv-140151/html} 上找到.}来解释设计 
Java 的初衷以及完成这些目标的情况. 接下来我们介绍其中的四个关键术语, 
分别是面向对象、健壮性、可移植性和解释性, 它们是现阶段你需要掌握的 Java 的四个核心特点.

\subsection{面向对象}

Java 是一种纯粹的面向对象编程语言, 这意味着一切皆为对象. 对象是程序的基本构建块, 具有状态和行为. 
相比于传统的面向过程编程, 面向对象编程使得代码更模块化, 易于维护和扩展. 在 Java 中, 
类和对象的概念是核心, 而且通过接口的灵活运用, 实现了对多继承的替代方案. 

\subsection{健壮性}

Java 的健壮性体现在多个方面. 首先, Java 强调编译期和运行时的错误检测, 
通过严格的语法检查和类型检查, 能够在程序执行前发现潜在问题. 其次, 
Java 的垃圾回收机制有助于防止内存泄漏, 大大提高了程序的稳定性. 此外, 
异常处理机制也使得开发者能够更好地处理错误情况, 确保程序在面对异常时依然能够稳健运行. 

\subsection{可移植性}

Java 的可移植性是通过字节码和 Java 虚拟机 (JVM) 实现的. Java 程序在编译时生成字节码, 
而不是直接生成机器码. 这种字节码可以在任何装有相应 Java 虚拟机的平台上运行. 
这使得 Java 程序无需重新编写即可在不同的操作系统和硬件上执行, 为跨平台开发提供了强大的支持. 

\subsection{解释性}

Java 是一种解释性语言, 它的程序在运行时通过 Java 虚拟机进行解释执行. 
这种特性为 Java 带来了很多优势. 首先, Java 程序可以逐行执行, 
无需预先编译成本地机器码, 使得开发和调试更为灵活. 
其次, 解释性也有助于实现平台无关性, 因为相同的字节码可以在不同平台上被解释执行, 无需修改源代码. 

\vspace*{1cm}

除此之外,
Java 在设计上注重安全性. 通过强大的安全管理器, Java 可以在运行时防止恶意代码的执行. 
Applet 的沙箱机制是其中的一例, 它限制了从网络上下载的代码对本地系统的访问权限, 
防止了潜在的安全威胁. 
Java 还拥有庞大而丰富的生态系统, 有着丰富的标准库和框架. 
这使得开发者可以轻松地利用现有的工具和资源来加速开发过程. 从企业级应用到移动应用, 
Java 生态系统涵盖了各种领域, 为开发者提供了广泛的选择和支持. 

\newpage
\thispagestyle{empty}

\part{Java 程序设计}

\chapter{变量与运算符}

\quad\quad 变量是赋给值的标签, 指向了内存中数据的存储单元. Java 也用变量来存储值.

\section{变量}
\section{数值数据类型及其操作}

\newpage
\section{数值数据类型变量的类型转换}

两个不同数值数据类型的操作数进行二元运算的时候, 或者把一个类型的变量赋值给另一个类型的变量
的时候, 就需要进行类型转换. 在 Java 中, 数值数据类型类型转换分为自动类型转换和强
制类型转换. \textbf{类型转换}是将一种数据类型的值转换成另一种数据类型的值的操作:
\begin{enumerate}[label=(\arabic*), itemsep=0pt]
    \item 从占用字节数较小的类型转换成占用字节数较大的类型的转换称为\textbf{扩大类型}
    (widening a type), 在 Java 中, 扩大类型一般是自动完成的;
    \item 从占用字节数较大的类型转换成占用字节数较小的类型的转换称为\textbf{缩小类型}
    (narrowing a type), 在 Java 中如果想要缩小类型, 就必须显式地进行类型转换.
\end{enumerate}

\subsection{自动类型转换}

在 Java 中, 数值数据类型之间的自动转换遵循\textbf{自动提升原则}, 
即在一个表达式中有多个数参与运算时, 先把这些参与运算的数转换为容量最大的类型, 
再计算表达式的结果. 怎么理解呢? 大烧杯能装的东西肯定比小烧杯能装的东西要多,
因此占用字节数更大的数据类型拥有更多的二进制位, 也就有更大的数据表示范围,
所以我们在计算时先把所有的数据都 “倒进最大容量规格的那一个烧杯中”.

\subsection{强制类型转换}

% \section{Java}

% \subsection{Java 的基本特性}
% \subsubsection{面向对象}
% \subsubsection{健壮性}
% \subsubsection{可移植性}
% \subsubsection{解释性}

% \part{Java 的面向对象特性}

% \chapter{Java 程序设计语言}

% Java 是一门面向对象的程序设计语言, 一个 Java 程序可以认为是一系列对象的集合, 这些对象通过调用彼此的方法来协同工作.

% \section{Java 基础语法}

% \begin{example}
%     我们从经典的 Hello World 程序开始我们的 Java 旅途, 并以它作为示例来介绍 Java 程序的结构和框架.
%     \begin{lstlisting}[language=java]
% // HelloWorld.java
% package com.example.helloworld;  
    
% public class HelloWorld {
%     public static void main(String[] args) {
%         System.out.println("Hello World!");  
%     }  
% }
%     \end{lstlisting}
% \end{example}


% 在上面的示例中, public 是访问修饰符, 用于控制程序的其他部分对这段代码的访问级别;
% static 关键字 
% void 返回类型说明 
% main 是方法名, Java 虚拟机总是从指定类中的 main 方法开始执行, main 方法的访问修饰符必须声名为 public.

% 关于 Java 的基础语法, 我们做几点说明:

% \begin{itemize}
%     \item 大小写敏感;
%     \item 类名应该采用 CamelCase 命名法, 即各个单词的首字母大写, 这种命名法的英文名很容易理解 —— 那些大写的字母就像驼峰一样凸起;
%     \item 方法名应该以小写字母开头, 第二个单词开始每个单词的首字母大写;
%     \item Java 源文件名必须和类名相同, 文件的后缀名为 .java, 因此 HelloWorld 的文件命名应为 HelloWorld.java;
%     \item 所有的 Java 程序都从 main 方法开始执行.
% \end{itemize}

% \chapter{封装性在 Java 的编程实现}

% % \section{Java Class}
% % \subsection{创建一个新的数据类型: class}

% % 在 Java 中, 一个类通过 class 关键字进行定义, 类的定义包括类声明和类体两个部分.

% % \subsubsection{类声明: 创建一个新的对象类型}

% % 在 Java 中声明一个类的语法如下:
% % \begin{lstlisting}[language=Java]
% % public class ClassName {
% %     private int data1;                      //成员变量声明
% %     ....
% %     public int method1(int arg1,....){
% %         ....
% %     }
% %     ....
% % }
% % \end{lstlisting}

% \section{使用预定义类}

% \subsection{对象与对象变量}

% 对象变量并不实际包含一个对象, 它只是引用了一个对象.

% \subsection{Java 类库中的 LocateDate 类}

% 事实上, 时间和度量和时间节点的命名是一件非常复杂的工作.

% \subsection{更改器方法与访问器方法}

% 访问并修改对象的方法称为\textbf{更改器方法} (mutater method), 只访问对象而不修改对象的方法称为\textbf{访问器方法} (accessor method).


% \begin{example}[当前月的日历]
%     编写程序显示当前月的日历, 如
%     \begin{lstlisting}
% Mon Tue Wed Thu Fri Sat Sun
%                           1 
%   2   3   4   5   6   7   8 
%   9  10* 11  12  13  14  15 
%  16  17  18  19  20  21  22 
%  23  24  25  26  27  28  29 
%  30  31
%     \end{lstlisting}
%     其中, “今天” 处要加星号.
%     \begin{lstlisting}[language=java]
% import java.time.*;

% public class CalendarTest
% {
%     public static void main(String[] args){
%         // 获取当前日期
%         LocalDate date = LocalDate.now();        
%         // 获取当前月份
%         int month = date.getMonthValue();        
%         // 获取当前日期的日
%         int today = date.getDayOfMonth();        

%         // 将日期设置为本月的第一天
%         date = date.minusDays(today - 1);        
%         // 获取本月第一天是星期几
%         DayOfWeek weekday = date.getDayOfWeek(); 
%         int value = weekday.getValue();          

%         // 打印星期标题
%         System.out.println("Mon Tue Wed Thu Fri Sat Sun");  
%         // 打印月份第一天之前的空格
%         for (int i=1; i < value; i++)
%             System.out.print("    ");            
        
%         // 循环打印月份的每一天
%         while (date.getMonthValue() == month) {  
%             // 使用printf格式化输出日期, 保持3位宽度
%             System.out.printf("%3d", date.getDayOfMonth()); 
%             // 如果日期是今天, 则在日期后面添加"*"
%             if (date.getDayOfMonth() == today)
%                 System.out.print("*");           
%             else
%                 System.out.print(" ");
%             // 将日期向后推移一天
%             date = date.plusDays(1);             
%             // 如果日期是星期一, 换行
%             if (date.getDayOfWeek().getValue() == 1)
%                 System.out.println();            
%         }
        
%         // 如果最后一天不是星期一, 换行
%         if (date.getDayOfWeek().getValue() != 1)
%             System.out.println();                   
%     }
% }        
%     \end{lstlisting}
% \end{example}

% \chapter{接口}

% \subsection{继承父类与实现接口的区别}

% \begin{lstlisting}
% package InterfaceLearning.hspInterface;

% public class ExtendsVsImplements {
%     public static void main(String[] args) {
%         LittleMonkey wuKong = new LittleMonkey("悟空");
%         wuKong.climbing();
%         wuKong.swimming();
%     }
% }

% class Monkey {
%     private String name;
%     public Monkey(String name) {
%         this.name = name;
%     }
%     public String getName() {
%         return this.name;
%     }
%     public void climbing() {
%         System.out.println(name+"会爬树....");
%     }
% }

% class LittleMonkey extends Monkey implements Fishable {
%     public LittleMonkey(String name) {
%         super(name);
%     }

%     @Override
%     public void swimming() {
%         System.out.println(this.getName()+"学会了游泳....");
%     }
% }

% interface Fishable {
%     void swimming();
% }
% \end{lstlisting}

% \begin{enumerate}
%     \item 当子类继承了父类, 就自动拥有了父类的功能
%     \item 如果子类需要扩展功能, 可以通过实现接口的方式扩展
%     \item 实现接口是对 Java 单继承机制的补充
%     \item 继承的主要价值在于: 解决代码的复用性和可维护性
%     \item 接口的价值在于: 设计好各种规范, 让其他类去实现这些方法, 使代码更加灵活
%     \item 接口比继承更加灵活
%     \item 继承 is-a
%     \item 接口 like-a
%     \item 接口在一定程度上实现代码解耦 (接口规范性+动态绑定)
% \end{enumerate}

% \part{Java 高级编程}

% \chapter{Java 多线程编程}

% \section{Java 线程}
% \subsection{中央处理器 (CPU) }

% \subsubsection{CPU 的组成}

% \begin{enumerate}[label=(\arabic*)]
%     \item \textbf{运算器}: 执行逻辑运算, 对数据进行加工处理: 
%     \item \textbf{控制器}: 负责协调并控制计算机各个部件执行程序的指令序列, 
%     包括取指令、分析指令和执行指令: 
%     \item \textbf{寄存器}: 保存运算或指令的一些临时数据: 
% \end{enumerate}

% \subsubsection{单核与多核 CPU}

% \begin{enumerate}[label=(\arabic*)]
%     \item \textbf{单核}: 在一个时间单元内只能执行一个任务, 同时间段内有多个任务需要 CPU
%     去执行时, CPU 通过交替去执行多个任务中的一个任务来实现, 但是由于其执行速度快, 
%     用户很难觉察: 
%     \item \textbf{多核}: 多核的 CPU 能同时执行多个任务. 
% \end{enumerate}

% \subsection{程序与进程}

% \subsubsection{程序}

% 一组计算机能够识别和执行的指令, 实现特定任务的运行. 

% 开发者所编写的代码称之为程序, 程序就是按照算法和数据结构构件的代码, 一组
% 数据和指令集, 是一个静态的概念. 

% \subsubsection{进程}

% \begin{definition}[进程]
%     应用程序在执行过程中占用的独立的系统资源 (内存、CPU、磁盘、网络) 称为\textbf{进程} (Process) . 
% \end{definition}

% \begin{enumerate}[label=(\arabic*)]
%     \item CPU 从硬盘中读取一段程序到内存中, 该执行程序的实例就叫做进程: 
%     \item 一个程序如果被 CPU 多次读取到内存中, 变成多个独立的进程, 将程序运行起来, 
%     我们称之为进程: 
%     \item 进程是程序的一次执行过程, 是一个动态的概念: 
%     \item 进程存在生命周期, 进程随着程序运行的终止而销毁. 
% \end{enumerate}

% \subsubsection{线程}

% \begin{definition}[线程]
%     运行于进程中的任务单元称为\textbf{线程} (Thread) , 是进程中实际运作的单位. 
% \end{definition}
% \begin{enumerate}[label=(\arabic*)]
%     \item 通常在一个进程中可以包含若干个线程: 
%     \item 线程是 CPU 调度和执行的最小单位: 
%     \item 一个进程可以有多个线程, 例如, 在 Bilibili 弹幕视频网看视频时可以同时看图像、听声音、看弹幕等: 
%     \item 很多线程都是模拟出来的, 真正的多线程是指有多个 CPU, 或多核 CPU, 如服务器, 
%     如果是模拟出来的多线程, 即一个单核 CPU 的情况下, 在一个时间点只能执行一个代码, 
%     因为切换很快, 所以有同时执行的错觉. 
% \end{enumerate}
% \begin{remark}
%     使用多线程机制之后, \lstinline|main()| 方法结束之时主线程结束了, 其他线程还没有结束, 但没有主线程也不能运行, 
%     最起码, 现在的 Java 程序中至少有两个线程并发, 一个是垃圾回收线程, 一个是执行 \lstinline|main()| 方法的主线程. 
% \end{remark}

% % \subsection{多线程开发}

% % \subsubsection{并发程序设计}

% % \begin{enumerate}[label=(\arabic*)]
% %     \item 同一对象被多个线程同时操作: 
% %     \item 特点: 同时安排多个任务, 这些任务可以彼此穿插着进行: 
% %     有些任务可能是并行的, 比如去菜市场买菜、回家发邮件和去洗浴中心学外语的某些路是重叠的, 
% %     这是的确是在做三件事: 
% %     但进菜市场买菜、回家发邮件和进洗浴中心学外语三者是互斥的, 每个时刻只能完成其中一件. 
% %     换句话说, 并发允许两个任务彼此干扰. 
% % \end{enumerate}

% % \subsubsection{并行}

% % 你做你的事、我做我的事, 互不干扰同时进行. 

% % \subsubsection{串行}

% % 一个程序处理房钱进程, 按顺序接着处理下一个进程, 一个接着一个进行. 

% \subsection{线程的生命周期}

% Java 中线程的生命周期分为以下几个状态:
% \begin{enumerate}
%     \item 新建状态 (new): 当线程对象被创建但还没有调用其 \lstinline|start()| 方法时, 线程处于新建状态. 在这个阶段, 线程仅仅是一个 Java 对象, 没有被分配到实际的操作系统线程. 
%     \item 就绪状态 (Runnble): 线程一旦调用了 \lstinline|start()| 方法, 它就进入就绪状态. 在就绪状态下, 线程等待被调度器分配一个可运行的时间片, 以便运行它的 \lstinline|run()| 方法. 
%     \item 运行状态 (Runing): 当线程被调度并执行了其 \lstinline|run()| 方法时, 它进入运行状态. 在运行状态中, 线程执行其实际的工作. 
%     \item 阻塞状态 (Bloked): 线程可能会在某些情况下被阻塞, 例如等待某个锁、等待输入/输出等. 当线程处于阻塞状态时, 它暂时停止执行, 并等待特定的条件发生, 使得线程可以继续执行. 
%     \item 等待状态 (Waiing): 线程进入等待状态是因为调用了 \lstinline|Object.wait()|、\lstinline|Thread.join()| 或类似的方法, 使线程等待某个条件的发生. 
%     \item 超时等待状态 (Timed Waiting): 类似于等待状态, 但在指定的时间内会自动返回, 例如通过调用 \lstinline|Thread.sleep()|、\lstinline|Object.wait(long timeout)| 或 \lstinline|Thread.join(long millis)| 等方法. 
%     \item 终止状态 (Terminted): 当线程的 \lstinline|run()| 方法执行完成或者线程调用了 \lstinline|Thread.stop()| 方法时, 线程进入终止状态. 一旦线程终止, 它就不能再重新启动. 
% \end{enumerate}

% 线程状态之间的转换如下: 

% - 新建状态 -> 就绪状态: 调用 `start()` 方法. 
% - 就绪状态 -> 运行状态: 被调度器选中并执行 `run()` 方法. 
% - 运行状态 -> 阻塞状态、等待状态、超时等待状态: 调用了相关的方法. 
% - 阻塞状态、等待状态、超时等待状态 -> 就绪状态: 条件满足. 
% - 运行状态 -> 终止状态: `run()` 方法执行完成. 
% - 阻塞状态、等待状态、超时等待状态 -> 终止状态: 线程被中断或执行完成. 


% \subsection{线程同步}

% 当有一个线程在对内存进行操作时, 其他线程都不可以对这个内存地址进行操作, 直到该线程完成操作,
% 其他线程才能对该地址进行操作, 而其他线程又处于等待状态, 实现线程同步的方法有很多种.

% 为什么要线程同步? 多个线程同步运行的时候可能调用线程函数, 在多个线程同时对同一个内存地址进行
% 写入, 由于 CPU 时间调度上的问题, 写入数据会被多次的覆盖, 所以需要使线程同步, 防止数据并发
% 导致的安全性问题.

\chapter{继承和多态}

面向对象范式将数据和方法结合在对象中, 使用面向对象范式的软件设计聚焦于对象以及对象上的
操作. 面向对象方法结合了面向过程范式的强大之处, 并且进一步将数据和操作集成在对象中.

继承对于软件重用是一个重要且功能强大的特征.

\chapter{抽象类和接口}

\section{抽象类}

抽象方法不能包含在非抽象类中, 如果抽象父类的子类不能实现所有的抽象方法, 那么子类
也必须定义为抽象的. 在继承自抽象类的非抽象子类中, 必须实现所有的抽象方法. 此外, 抽象方法
是非静态的.

抽象类不能使用 \lstinline|new| 操作符来初始化, 但仍然可以定义它的构造方法, 这个构造方法
在其子类的构造方法中调用.

包含抽象方法的类必须是抽象的, 但可以定义一个不包含抽象方法的抽象类, 这个抽象类用作
定义新子类的基类.

子类可以重写父类的方法并将其定义为抽象的, 这是很少见的一种做法, 一般用于父类的方法实现
在子类中变得无效的情况. (尝试举例?)

即使父类是具体类, 子类也可以是抽象的.

不能使用 \lstinline|new| 操作符来从一个抽象类创建一个实例, 但是抽象类可以用作一种数据类型.

\section{接口}

接口的目的是指明相关或者不相关的类的对象的共同行为. 定义接口的语法为
\begin{lstlisting}[language=Java]
accessModifier interface InterfaceName {
    
}
\end{lstlisting}

在 Java 中, 接口可以看作是一种特殊的类, 你可以使用接口作为引用变量的数据
类型或者作为类型转换的结果类型, 但是不能使用 \lstinline|new| 操作符创建
接口的实例.

一个类可以使用 \lstinline|implements| 关键字来实现一个接口.

接口中修饰数据域的 \lstinline|public static final| 和修饰方法的
\lstinline|public abstract| 可以被忽略, 但是在实现该接口的类中,
必须将接口中的方法声明为 \lstinline|public| 的.

如果在声明接口方法时使用了关键字 \lstinline|default|, 那么你实际上
给出了一个默认的接口方法, 默认的接口方法需要在接口的定义中实现. 实现该接口
的类可以简单地使用默认的方法来实现, 或者重写该接口方法. 利用该特征,
在一个具有默认实现的已有接口中添加新的方法时, 只要声明默认方法并给出实现,
就可以不必为已经实现了该接口
的已有类重新编写接口方法的实现.

接口中的公共静态方法与类中的公共静态方法相同.

在实现接口的默认方法和公共静态方法时, 可以定义并调用接口的私有方法.

\begin{table}[H]
    \centering
    \caption{接口中的方法}
    \begin{tabular}{p{0.5\textwidth}p{0.45\textwidth}}
        \hline 
        \textbf{方法声明部分形如……} & \textbf{使用方法} \\ 
        \hline 
        \lstinline|public abstract returnType p()| & 等价于 \lstinline|returnType p()| \\
        \lstinline|public default returnType p()| & 默认接口方法, 实现该接口的子类可以选择直接使用它或者重写它 \\ 
        \lstinline|public static returnType p()| & 公共静态方法, 与类中的公共静态方法的使用方式相同 \\ 
        \lstinline|private returnType p()| & 私有方法, 可以被默认接口方法和公共静态接口方法调用 \\ 
        \hline
    \end{tabular}
\end{table}

\section{标记接口}

内部为空的接口称为\textbf{标记接口} (marker interface),
标记接口既不包括敞亮也不包括方法, 它用来表示一个类拥有某些希望具有的特征.

\section{抽象类和接口的区别与联系}

\begin{table}[H]
    \centering
    \caption{抽象类与接口}
    \begin{tabular}{cp{0.3\textwidth}p{0.25\textwidth}p{0.3\textwidth}}
        \hline 
        & \textbf{变量} & \textbf{构造方法} & \textbf{方法} \\
        \hline
        抽象类 & 无限制 & 子类通过构造方法链调用构造方法, 抽象类不能用 \lstinline|new|
        操作符实例化 & 无限制 \\ 
        接口 & 所有的变量必须是公开、静态、最终, 即 \lstinline|public static final| 的 &
        没有构造方法, 不能用 \lstinline|new| 操作符实例化 
        & 可以包含 \lstinline|public abstract| 的抽象实例方法、\lstinline|public default|
        的默认接口方法以及 \lstinline|public static| 的静态方法. 在实现默认接口方法
        与静态方法时, 作为辅助, 可以定义并调用 \lstinline|private| 的接口内部私有方法. \\ 
        \hline
    \end{tabular}
\end{table}

\part{数据结构与算法}

\chapter{线性表}

\section{链表}

链表是一种数据结构, 其中的元素 (节点) 通过指针相互连接. 本节我们介绍如何在 Java 中实现一个
简单的链表. 它主要有三个类构成 —— 实现节点的 \lstinline|Node| 类、
实现链表本身的 \lstinline|LinkedList| 类, 以及一个用于测试的编写了 \lstinline|main| 方法的测试类.

\subsection{\lstinline|Node| 类: 链表的节点}

% \lstinputlisting[language=Java]{../IdeaProjects/LearningJava/src/LinkedList/Node.java}

在类 \lstinline|Node| 中, 我们定义了两个属性: \lstinline|data| 属性表示我们存放在这个
节点中的数据的值, 而 \lstinline|next| 属性指的是当前节点的下一节点. 为了保证程序的封装性
不被破坏, 你可以把这些参数设置为私有 (\lstinline|private|), 但现在我们简单起见, 把这些
属性的访问修饰符都设置为缺省, 这意味着我们可以通过简单的赋值运算实现属性的访问和变更, 而不需要
使用公共的 \lstinline|getter| 和 \lstinline|setter|.

我们为 \lstinline|Node| 类提供了一个构造器, 它的作用是把节点的 \lstinline|data| 属性
设置为我们给定的形式参数 \lstinline|data|, 然后默认它是链表的尾节点, 即 \lstinline|this.next = null|.

\subsection{\lstinline|LinkedList| 类: 链表的实现}

我们先给出 \lstinline|LinkedList| 类的完整的实现:
% \lstinputlisting[language=Java]{../IdeaProjects/LearningJava/src/LinkedList/LinkedList.java}

\part{代码设计}

\chapter{代码设计概述}

学习代码设计的相关知识可以帮助我们写出可扩展、可读和可维护的高质量代码.

{\captionof{table}{代码质量的评判标准} % 表格标题
\label{代码质量的评判标准} % 交叉引用标签
\begin{longtable}{cp{0.4\textwidth}p{0.4\textwidth}}
    \toprule
    % 表头
    \textbf{特征} & \textbf{含义} & \textbf{评判标准} \\
    \midrule
    \endhead
    \bottomrule
    \endfoot

    % 表格内容
    可维护性 
    & 在不破坏原有代码设计、不引入新 bug 的情况下, 能够快速修改或添加代码 
    & 与多种因素有关: 代码简洁、可读性好、可扩展性好、分层清晰、模块化程度高、
    高内聚低耦合、遵守基于接口而非实现编程的设计原则 \\
    可读性
    & 代码是否易读、易理解
    & 代码符合代码规范、命名达意、注释详尽、函数长度合理、模块划分清晰、高内聚低耦合、Code Review \\ 
    可扩展性
    & 在不修改或少量修改原有代码的情况下, 能够通过扩展方式添加新功能
    & \\ 
    灵活性 & 含义比较宽泛 & \\ 
    简洁性 & 代码简单、逻辑清晰 & KISS 原则 \\ 
    可复用性 & 尽量减少重复代码的编写, 复用已有代码 & DRY 原则 \\ 
    可测试性 & 易于编写单元测试 & \\
\end{longtable}}

\section{关于面向对象编程范式的进一步讨论}

面向对象编程范式因其丰富的特性 (封装、抽象、继承和多态) 可以实现很多复杂的设计思路,
因此它是很多设计原则、设计模式编码实现的基础.

\subsection{面向对象编程范式的特性}

\subsubsection{封装 (encapsulation)}
 
封装也称为信息隐藏或数据访问保护. 类通过暴露有限的访问接口, 授权外部仅能通过类
提供的方式来访问内部信息或数据. 这看似影响了程序代码的灵活性, 但其实过度灵活意味着
不可控和不安全, 属性可以通过各种奇怪的方式被随意修改, 而且修改逻辑可能散落在代码的
各个角落, 进而影响代码的可读性和可维护性.

类通过提供有限的方法暴露必要的操作, 能提高类的易用性. 将属性封装, 暴露少许的必要的方法
给调用者, 调用者就不需要了解太多的业务细节, 用错的概率也会降低很多.

\subsubsection{继承 (inheritance)}

继承最大的作用是代码复用.

\subsubsection{多态 (polymorphism)}

多态是指, 在代码运行过程中我们可以用子类替换父类, 并调用子类的方法.

\newpage
\thispagestyle{empty}

\chapter{设计模式}

\section{创建型设计模式}

\subsection{单例模式}

% \begin{code}
%     \lstinline|Strik0r.java|
%     \lstset{
%         moreemph={INSTANCE},
%         emphstyle=\color{class_field},
%     }
%     \lstinputlisting[language=Java]
%     {../IdeaProjects/Hello/src/cn/edu/nwpu/software/strik0r/root/Strik0r.java}
% \end{code}
\part{编程速查}

\chapter{时间相关的需求……}

\section{\lstinline|Calendar| 类及其子类: 表示精确到毫秒的特定时刻}

先前我们曾经介绍过, 学习各种 Java 类的核心在于明确该类在对哪一种客观物质世界
的对象进行建模. 一个具体的 \lstinline|GregorianCalendar| 对象代表了一个具体
的公历 (gregorian calendar) 时间节点.

{\captionof{table}{\lstinline|Calendar| 类的域常量} % 表格标题
% \label{} % 交叉引用标签
\begin{longtable}{p{0.3\textwidth}p{0.5\textwidth}}
    \toprule
    % 表头
    \textbf{常量} & \textbf{说明} \\

    \toprule
    \endhead
    \bottomrule
    \endfoot

    % 表格内容
    \lstinline|YEAR| & 年份 \\ 
    \lstinline|MONTH| & 月份, 其中 $0$ 表示 $1$ 月 \\ 
    \lstinline|DATE| & 日前 \\ 
    \lstinline|OUR| & 小时 (12 小时制) \\ 
    \lstinline|HOUR_OF_THE_DAY| & 小时 (24 小时制) \\ 
    \lstinline|MINUTE| & 分钟 \\ 
    \lstinline|SECOND| & 秒 \\ 
    \lstinline|DAY_OF_WEEK| & 一周中的哪一天, 1 表示星期天 \\ 
    \lstinline|DAY_OF_MONTH| & 同 \lstinline|DATE| \\ 
    \lstinline|DAY_OF_YEAR| & 一年中的哪一天, 其中 1 表示该年的第一天 \\ 
    \lstinline|WEEK_OF_MONTH| & 一月中的哪一周, 其中 1 表示该月的第一周 \\ 
    \lstinline|WEEK_OF_YEAR| & 一年中的哪一周, 其中 1 表示该年的第一周 \\ 
    \lstinline|AM_PM| & $0$ 表示上午, $1$ 表示下午 \\ 
\end{longtable}}

\subsection{复习参考题}

\begin{exercise}
    可以使用 \lstinline|Calendar| 类来创建一个 \lstinline|Calendar| 对象吗?
    这显然是不可以的, 因为 \lstinline|Calendar| 类是一个抽象类, 不能够
    通过 \lstinline|new| 操作符构造一个 \lstinline|Calendar| 对象.
\end{exercise}

\input{LateX_tem/appendix.tex}

\newpage
\begin{thebibliography}{1}
    \addcontentsline{toc}{chapter}{参考文献}
    \bibitem{核心技术1}
    [美] Cay S. Horstmann. 
    Java 核心技术: 原书第 12 版. 卷 I: 开发基础[M].
    林琪, 苏钰涵译.
    北京: 机械工业出版社, 2022
    \bibitem{核心技术2}
    [美] Cay S. Horstmann.
    Java 核心技术: 原书第 12 版. 卷 II: 高级特性[M].
    陈昊鹏译.
    北京: 机械工业出版社, 2023
    \bibitem{设计模式之美}
    王争. 设计模式之美 [M]. 北京: 人民邮电出版社, 2022.
\end{thebibliography}

\input{LateX_tem/endpage.tex}

\end{document}